\documentclass[letterpaper]{article}

\usepackage[utf8]{inputenc}
\usepackage{geometry}
\usepackage{url}
\usepackage{hanging}
\usepackage{parskip}
\usepackage[compact]{titlesec}
\usepackage[warn]{textcomp}
\usepackage{pifont}
\usepackage{natbib, bibentry}
\makeatletter\let\saved@bibitem\@bibitem\makeatother

\usepackage{hyperref}
\makeatletter\let\@bibitem\saved@bibitem\makeatother

% Comment the following lines to use the default Computer Modern font
% instead of the Palatino font provided by the mathpazo package.
% Remove the 'osf' bit if you don't like the old style figures.
%\usepackage[T1]{fontenc}
%\usepackage[sc,osf]{mathpazo}

\pagestyle{empty}

% Set your name here
\def\name{Yang Wang}

% Replace this with a link to your CV if you like, or set it empty
% (as in \def\footerlink{}) to remove the link in the footer:
\def\footerlink{}

% The following metadata will show up in the PDF properties
\hypersetup{
  colorlinks = true,
  urlcolor = black,
  pdfauthor = {\name},
  pdfkeywords = {},
  pdftitle = {\name: Curriculum Vitae},
  pdfsubject = {Curriculum Vitae},
  pdfpagemode = UseNone
}
\hypersetup{pdfstartview={XYZ null null 1.00}}

\geometry{
%  body={6.5in, 8.5in},
  left=0.35in,
  right =0.4in,
  top=0.3in,
  bottom=0.3in,
}

% Customize page headers
%\pagestyle{myheadings}
%\markright{\name}
%\thispagestyle{empty}

% Custom section fonts
\usepackage{sectsty}
\sectionfont{\rmfamily\bfseries\large}
%\subsectionfont{\rmfamily\mdseries\itshape\large}

% Other possible font commands include:
% \ttfamily for teletype,
% \sffamily for sans serif,
% \bfseries for bold,
% \scshape for small caps,
% \normalsize, \large, \Large, \LARGE sizes.

% Don't indent paragraphs.
\setlength\parindent{0em}

% Make lists without bullets
%\renewenvironment{itemize}{
%  \begin{list}{}{
%    \setlength{\leftmargin}{1.5em}
%  }
%}{
%  \end{list}
%}
\renewcommand{\labelitemii}{\ding{226}}

\begin{document}

% Place name at left
%{\huge \name}

% Alternatively, print name centered and bold:
\centerline{\LARGE \bf \name}

%\vspace{0.25in}

\begin{minipage}{\linewidth}
%\href{http://*}{}
\begin{center}
Cambridge, MA, 02139, United States\\
Email: \href{mailto:guliyolanda@gmail.com}{\tt guliyolanda@gmail.com} , Tel: +1(617)7109267 \\
Homepage:  \href{https://guliyolanda.github.io}{\tt https://guliyolanda.github.io} \\
\end{center}
\end{minipage}

\vspace{0pt}
\rule{\textwidth}{1pt}
%\ul{\Huge\textbf{Some text\mbox{\hspace{1in}}}}

\vspace{-12pt}
\section*{Research Interests}
\vspace{-10pt}
\begin{itemize}
\item Computational Fluid Dynamics in multi-disciplinary applications
\item Scientific computation, multi-physics modelling and optimisation
\item Energy (generation, storage, and utilization) system modelling and optimization
%\item Numerical modeling and optimisation
%\item Data-driven modelling and machine learning 

\end{itemize}

\vspace{-12pt}
\section*{Skills}
\vspace{-10pt}
\begin{itemize}
\item Numerical methods for Partial Differential Equations: algorithms, discretisation schemes and linear solvers
\item Multi-scale and multi-physics modelling, numerical optimisation: stochastic and deterministic methods
%\item Finite Volume Method and Spectral Element Method (Finite Element Method + Spectral Method)
%\item Numerical optimisation: stochastic and deterministic methods
\item Fortran, MATLAB/SIMULINK and C++ programming
%\item Chemical Engineering process/system simulation tool: 
\item Tools/software: GMSH, ICEM, ParaView, OpenFOAM, Fluent%, Aspen Plus, SolidWorks, ProE
\item Automatic Differentiation (AD) tool TAPENADE
\item Linux operation systems, Shell script, Vim, Meld, Make, Git, Doxygen, Gprof, and HYPRE open source libraries
\item MS Office tools (e.g. Word, Excel and PowerPoint), Texmaker (editor for \LaTeX), etc.
\end{itemize}

\vspace{-12pt}
\section*{Education}
\vspace{-10pt}
\begin{itemize}
  \item 2012. 9 - 2017. 1 \hspace{2pt} PhD in Mechanical Engineering, Queen Mary, University of London, United Kingdom
  \item 2009. 9 - 2012. 6 \hspace{2pt} MSc in Power Engineering and Thermophysics, Xi’an Jiaotong University, China
  \item 2005. 9 - 2009. 7 \hspace{2pt} BEng in Energy and Power Engineering, Xi’an Jiaotong University, China
\end{itemize}

\vspace{-12pt}
\section*{Work/Research Experience}
\vspace{-10pt}
\begin{itemize}
\item 2018. 12 - present \hspace{2pt} Flight Data Analyst Intern at Altaeros, Somerville, Massachusetts, United States
	\begin{itemize}
	\item Post-processing day-to-day flight field test data including aerostat motion, environmental and diagnosis data
	\item Analysing flight performance under environment variations
	\item Developing automatic data processing/analysing tools and maintaining codebase and improving robustness (in Matlab/SIMULINK and Batch Script)
	\item Cooperating with mechanical, electrical, control and other teams to design aerostats
	\end{itemize}
\item 2017. 11 - 2018. 12 \hspace{2pt} Parenting and working from home during relocation, Cambridge, USA
		\begin{itemize}
		\item Studying data types/structures and algorithms in C++
		\item Researching on multi-physics modelling and computation strategies
		\item Machine Learning by Stanford University on Coursera
		\item Writing and publishing journal papers; doing peer reviews for journals
		\end{itemize}		
\item 2017. 3 - 2017. 10  \hspace{2pt} Research fellow on projects funded by Engineering and Physical Science Research Council (EPSRC), School of Engineering, University of Warwick, Coventry, United Kingdom
		\begin{itemize}
		\item \textsl{	Next Generation Grid Scale Thermal Energy Storage Technologies}: 
					\begin{itemize}
					\item Developed a MATLAB/SIMULINK toolbox with the team for designing, controlling and optimising grid-scale thermal energy storage system% (sensible, phase-change, packed bed, and heat reservoir) for optimally designing heat storage systems;
%					\item Developed dynamic models of various components in compressed air energy storage (CAES) and thermal energy storage (TES), including mechanical components (piston, screw, scroll, centrifugal and axial compressors/expanders), heat exchangers, and air reservoirs in MATLAB/SIMULINK. These models are used for system design, control and optimization;
					%\item Using the developed model, various applications of energy storage were carried out, including optimal design of packed bed thermal energy storage, optimal design of radial turbine in CAES applications, system optimization of grid-scale CAES and TES applications, etc.
					\item Evaluated system performance by assembling and analysing the time-dependent and multi-scale simulation data set in many case studies of energy systems
					\item Improved the system cycle efficiency by \textasciitilde 10\% by optimizing the storage materials in the heat storage
					\end{itemize}

		\item	\textsl{Ultra-Supercritical (USC) steam power generation technology with Circulating Fluidized Bed (CFB): Combustion, Materials and Modelling}: 
					\begin{itemize}
					\item Created a light-weight super-fluid property data package (MATLAB and Fortran), which is an extendible, high speed, and platform-neutral tool for academic and industrial partners in the project
					\item Developed a data transfer protocol for the heat exchange coupling between the CFB boiler and the the water-wall, to computational-effectively increase fidelity of the system-level modelling
					\item Built a realistic numerical prototype via collecting and analysing data from industrial partners and collaborating with the Tsinghua University for model comparison and validation
					%\item Built the super-fluid property data package for Ultra-Supercritical water/steam system
					%\item Developed heat transfer numerical models of a CFB boiler integrated with water-wall heat exchange
					\end{itemize}
		\end{itemize}

\item 2012. 9 - 2017. 1  \hspace{2pt} PhD candidate, working on the project \textsl{About Flow} funded by the European commission
		\begin{itemize}
		\item Led the development team of in-house CFD codes (in Fortran) for incompressible flow with discrete adjoint sensitivity/gradient solvers using Automatic Differentiation
		\begin{itemize}
		\item Developed solvers on cell-centered/face-based and node-centered/edge-based data structure
		\item Preprocessed mesh data: sorting, listing, searching and tagging elements and calculating geometric information
		\item Restructured high-order schemes via node data interpolations
		\item Proposed the pseudo-inverse approach for Pressure Schur Complement (non-linear flow system algorithms) to solve incompressible Navier-Stokes Equation
		\item Applied matrix preconditioning techniques for solving discretised linear sub systems with data compressed in Compressed Raw Storage (CRS)
		%Implemented interfaces for data passing through Compressed Raw Storage (CRS)
%		
%		\item Applied matrix preconditioning techniques for solving discretised linear sub systems
		\item Applied data post-processing and visualization in Matlab and implemented interfaces (in Fortran) to .vtk output files compatible with ParaView 
		%\end{itemize}
		%\item	SIMPLE-like algorithms vs. Pressure Schur Complement (PSC) method theoretical derivation
		%\begin{itemize}
		\item Proposed the pseudo-inverse approach for Pressure Schur Complement (non-linear flow system algorithms) to solve incompressible Navier-Stokes Equation
		\item Applied matrix preconditioning techniques for solving discretised linear sub systems
		\item Improved the solution accuracy with residuals reduced by 1 order of magnitude
		\item Reduced CPU time by \textasciitilde 60\% in a run of both flow and gradient computation (on PCs and the HPC cluster)
		\item Increased the solver robustness for convergence in wider control parameter space and skewer mesh cases
		\item Expanded the solver compatibility to other data post-processing tools for better analysing and visualising data
		\end{itemize}
		\item CAD-based shape optimization of the S-bend air duct in Volkswagen Golf vehicle for reducing pressure drop
		\begin{itemize}
%		Processed the 3D mesh perturbation using NURBS-based parametrisation with continuity constraints for gradient validation of surface nodes w.r.t. control points
		\item Processed the 3D mesh perturbation using in-house CAD tool (NURBS-based parametrisation with continuity constraints) for gradient calculation of surface nodes w.r.t. control points
		\item	Developed CAD-based shape optimization driver (in Fortran and Shell script)
		\item Optimised the shape of S-bend air duct and achieved pressure loss reduction by \textasciitilde 20\%%analysed fluid dynamics at different inlet speeds%with mesh deformation based on linear elasticity theory
		\end{itemize}
		%Gradient validation on 
		
		
		%\item	Fluid dynamics analysis on  shape optimisation cases 
		\item	Node-based shape optimisation of the filaments in membrane channels for reducing pressure drop and increasing mass transfer rate
		%Developed and implemented numerical models for Pressure Retarded Osmosis (PRO) and Reverse Osmosis (RO) membrane process
					\begin{itemize}
					\item Developed and implemented numerical models for Pressure Retarded Osmosis (PRO) and Reverse Osmosis (RO) membrane process (in OpenFOAM and Fortran codes)
%					\item Validated the numerical models with experiment data
					\item Computed mass transport across the membrane in dual-channel flow using OpenFOAM preprocessing tools
					
					%\item Implemented particle swarm optimization algorithm, differential evolutionary optimization algorithm, non-dominated sorting genetic algorithm II (NSGA-II) in MATLAB
					%\item Applied the MATLAB-based model library and the implemented optimization algorithms to various applications of system optimization, e.g. optimal operation of desalination system, system design of solar powered desalination, etc.
					\item Firstly analysed filament surface sensitivities obtained from discrete adjoint computation
					\item Designed/Optimised the spacer shape (in Fortran and Shell script) and achieved pressure loss reduction by \textasciitilde 25\% with negligible mass transfer loss					
					% via gradient-based optimisation driver using discrete adjoint approach and analysed fluid dynamics and filament surface sensitivities
					\item Developed a membrane model library (in MATLAB) for simulating the flow and mass transfer of water and salts in desalination membrane processes
					\end{itemize}
		
		\end{itemize}
		
\item 2009. 9 - 2012. 7  \hspace{2pt} Postgraduate researcher at School of Energy and Power Engineering, Xi’an Jiaotong University, China
\begin{itemize} 
\item Spectral element method for acoustic propagation problem in non-uniform flows
		\begin{itemize}
		\item Studied Spectral Element Method, the combination of Finite Element and Spectral discretisation methods methods for high-accuracy, multi-scale flow and acoustic coupling computation% for Mathematical Physics Equations
		\item	Firstly derived the mathematical description of acoustic propagation in non-uniform flow
		\item	Implemented governing equation and %group velocity method with 
		high accuracy on the absorbing boundary conditions (in C++)
		\end{itemize}
\end{itemize}

\item 2008. 9 - 2009. 6  \hspace{2pt} Undergraduate research project: the design of high flow rate vortex/generative blower
		\begin{itemize}
		\item Impeller design based on empirical correlations in literature and 3D model via software ProE
		\end{itemize}
		
\end{itemize}

\vspace{-12pt}
\section*{Selected Publications}
\vspace{-10pt}

%\printbibliography[env=numbered+bold, heading=none, sorting=ynt, resetnumbers=true]

%\def\FormatName#1{%
%  \def\myname{Yang Wang}%
%  \edef\name{#1}%
%  \ifx\name\myname
%    \textbf{#1}%
%  \else
%    #1%
%  \fi
%}

\nobibliography{publications_yw, abbrev}
\bibliographystyle{abbrv}

\begin{enumerate}
\item \bibentry{wang18stablisation}
\item \bibentry{WANG201926}
\item \bibentry{wang18CAES}
\item \bibentry{HE20172120}
\item \bibentry{wang2016computational}
\item \bibentry{zhang2016geometric}
\item \bibentry{he2016evaluation}
\item \bibentry{geng2016research}
\item \bibentry{akbarzadeh2015fixed}
\item \bibentry{wang2015stabilisation}
\item \bibentry{he2015maximum}
\item \bibentry{he2015stand}

%\item \bibentry{}
\end{enumerate}


\vspace{-12pt}
\section*{Awards and grants}
\vspace{-10pt}
\begin{itemize}
\item 2015 Postgraduate Research Fund (Queen Mary University of London)
\item 2014 Postgraduate student grant (School of Engineering and Material Science, QMUL)
\item 2012 Best Postgraduates (Top 10\%)
\item 2010 Outstanding Postgraduate Student Award (Top 15\%)
\item 2009 Postgraduate Innovation Fund Scholarship (1st Class, 2/46)
\item 2009 Best Graduates (Top 10\%)
\item 2008 \textsl{Fusheng} Industrial Scholarship (1st Class, Top 15\%)
\end{itemize}

\vspace{-12pt}
\section*{Teaching and supervising experiences}
\vspace{-10pt}

\begin{itemize}
\item	2012 - 2015  \hspace{2pt} Teaching and demonstrating in undergraduate courses:
			\begin{itemize}
			\item Heat Transfer and Fluid Mechanics: coursework tutorial
			\item Mechanics of Fluids: lab demonstration and reports marking
			\item Computer Aided Engineering in Fluids and Solids: OpenFOAM tutorial
			\end{itemize}

%		\begin{itemize}
%		\item	Heat transfer and Fluid Mechanics%: teaching assistant
%		\item	Mechanics of Fluids%: experiment demonstration and report marking
%		\item	Computer Aided Engineering in Fluids and Solids%: OpenFOAM tutorial
%		\end{itemize}
\item 2012 - 2015 \hspace{2pt} Leader of the segregated flow solver development team:
		\begin{itemize}
		\item Mentoring junior researchers with code review and implementation
		\end{itemize} 
\item	2009 - 2012  \hspace{2pt} Instructor in Department of Fluid Machinery
\end{itemize}

% Footer
%\begin{center}
%  \begin{footnotesize}
%    Last updated: \today \\
%  \end{footnotesize}
%\end{center}

\end{document}
