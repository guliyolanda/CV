\documentclass[letterpaper]{article}

\usepackage[utf8]{inputenc}
\usepackage{geometry}
\usepackage{url}
\usepackage{hanging}
\usepackage{parskip}
\usepackage[compact]{titlesec}
\usepackage[warn]{textcomp}
\usepackage{pifont}
\usepackage{natbib, bibentry}
\makeatletter\let\saved@bibitem\@bibitem\makeatother

\usepackage{hyperref}
\makeatletter\let\@bibitem\saved@bibitem\makeatother

% Comment the following lines to use the default Computer Modern font
% instead of the Palatino font provided by the mathpazo package.
% Remove the 'osf' bit if you don't like the old style figures.
%\usepackage[T1]{fontenc}
%\usepackage[sc,osf]{mathpazo}

\pagestyle{empty}

% Set your name here
\def\name{Dr Yang Wang}

% Replace this with a link to your CV if you like, or set it empty
% (as in \def\footerlink{}) to remove the link in the footer:
\def\footerlink{}

% The following metadata will show up in the PDF properties
\hypersetup{
  colorlinks = true,
  urlcolor = black,
  pdfauthor = {\name},
  pdfkeywords = {},
  pdftitle = {\name: Curriculum Vitae},
  pdfsubject = {Curriculum Vitae},
  pdfpagemode = UseNone
}
\hypersetup{pdfstartview={XYZ null null 1.00}}

\geometry{
%  body={6.5in, 8.5in},
  left=0.35in,
  right =0.4in,
  top=0.3in,
  bottom=0.3in,
}

% Customize page headers
%\pagestyle{myheadings}
%\markright{\name}
%\thispagestyle{empty}

% Custom section fonts
\usepackage{sectsty}
\sectionfont{\rmfamily\bfseries\large}
%\subsectionfont{\rmfamily\mdseries\itshape\large}

% Other possible font commands include:
% \ttfamily for teletype,
% \sffamily for sans serif,
% \bfseries for bold,
% \scshape for small caps,
% \normalsize, \large, \Large, \LARGE sizes.

% Don't indent paragraphs.
\setlength\parindent{0em}

% Make lists without bullets
%\renewenvironment{itemize}{
%  \begin{list}{}{
%    \setlength{\leftmargin}{1.5em}
%  }
%}{
%  \end{list}
%}
\renewcommand{\labelitemii}{\ding{226}}

\begin{document}

% Place name at left
%{\huge \name}

% Alternatively, print name centered and bold:
\centerline{\LARGE \bf Dr Yang Wang}
\centerline{\large Senior Modelling Consultant – Application}
\vspace{0.1in}

%\vspace{0.25in}

\begin{minipage}{\linewidth}
%\href{http://*}{}
\begin{center}
Email: \href{mailto:guliyolanda@gmail.com}{\tt guliyolanda@gmail.com} , Tel: 07784670342 \\
Homepage:  \href{https://guliyolanda.github.io}{\tt https://guliyolanda.github.io} \\
\end{center}
\end{minipage}

\vspace{0pt}
\rule{\textwidth}{1pt}
%\ul{\Huge\textbf{Some text\mbox{\hspace{1in}}}}

\vspace{-12pt}
\section*{Interests}
\vspace{-10pt}
\begin{itemize}
\item Whole energy system modelling and optimisation for Net Zero transition
\item Strategic energy planning and stakeholder engagement
\item Innovation in sustainable energy technologies, including the potential of AI and data-driven methods
\end{itemize}

\vspace{-12pt}
\section*{Key Skills}
\begin{itemize}
\item Strategic energy planning tool development and deployment (ESME, ESME-Flex, EPN, LAEPForward, etc.)
\item Stakeholder engagement and relationship management (internal and external)
\item Interpretation and communication of modelling outputs for decision-making
\item Project leadership, delivery, and line management
\item Data analysis, scenario modelling, and reporting
\item Programming: Python, MATLAB/SIMULINK, Fortran; Version control (Git, SVN, Mercurial)
\end{itemize}

\vspace{-12pt}
\section*{Education}
\vspace{-10pt}
\begin{itemize}
  \item 2012 - 2016 \hspace{2pt} PhD in Mechanical Engineering, Queen Mary, University of London, United Kingdom
  \item 2009 - 2012 \hspace{2pt} MSc in Power Engineering and Thermophysics, Xi’an Jiaotong University, China
  \item 2005 - 2009 \hspace{2pt} BEng in Energy and Power Engineering, Xi’an Jiaotong University, China
\end{itemize}

\vspace{-12pt}
\section*{Work/Research Experience}
\vspace{-10pt}
\begin{itemize}
\item 2021.1 - present \hspace{2pt} Senior Modelling Analyst at Energy Systems Catapult, Birmingham, United Kingdom
  \begin{itemize}
    \item Led the development tasks, deployment, and interpretation of strategic energy planning tools (ESME, ESME-Flex, EPN, LAEPForward, etc.) for national and local energy system modelling.
    \item Interpreted and communicated modelling outputs to inform strategic decision-making for clients and project partners.
    \item Provided technical leadership and advice to project teams; mentored junior analysts.
    \item Delivered high-impact projects (e.g., ITEG, LAEPs, Cogen, EFM, ITNZ3, etc.) supporting Net Zero strategies.
    \item Supported business development by leveraging academic connections and building partnerships with institutions including Warwick Energy Innovation Centre and King's College London.
  \end{itemize}
\item 2019. 10 - 2020.12 \hspace{2pt} CFD Software Engineer (solver developer) at Advanced Design Technology, London, United Kingdom
   \begin{itemize}
   \item High performance turbo-machinery design tools: TDShaper and TD1
   \end{itemize}
\item 2018. 12 - 2019. 9 \hspace{2pt} Flight Data Analyst at Altaeros(MIT spin-out), Somerville, Massachusetts, United States
	\begin{itemize}
	\item Developing data tools,  processing and analysing day-to-day flight and telecom field test data and work with the mechanical, electrical, control teams to optimise the design of aerostats
% 	for including aerostat motion, environmental and diagnosis data
% 	\item Analysing flight performance under environment variations
%	\item Developing automatic data processing/analysing tools and maintaining codebase and improving robustness
	%(in Matlab/SIMULINK and Batch Script)
% 	\item Cooperating with mechanical, electrical, control and other teams to design aerostats
	\end{itemize}
\item 2017. 11 - 2018. 12 \hspace{2pt} Parenting and working from home during relocation, Cambridge, USA
		\begin{itemize}
% 		\item Studying data types/structures and algorithms in C++
% 		\item Researching on multi-physics modelling and computation strategies
		\item Machine Learning taught by Stanford University on Coursera
%		\item Writing and publishing journal papers %doing peer reviews for journals
		\end{itemize}		
\item 2017. 3 - 2017. 10  \hspace{2pt} Research fellow on projects funded by Engineering and Physical Science Research Council (EPSRC), School of Engineering, University of Warwick, Coventry, United Kingdom
		\begin{itemize}
		\item Next Generation Grid Scale Thermal Energy Storage Technologies
%					\begin{itemize}
%					\item Developed a MATLAB/SIMULINK toolbox with the team for designing, controlling and optimising grid-scale energy storage and integrated energy system, such as compressed air energy storage, battery storage, and thermal energy storage% (sensible, phase-change, packed bed, and heat reservoir) for optimally designing heat storage systems;
%					%\item Developed dynamic models of various components in compressed air energy storage (CAES), thermal energy storage (TES) and combined cycle gas turbine (CCGT) systems%, including mechanical components (piston, screw, scroll, centrifugal and axial compressors/expanders), heat exchangers, and air reservoirs in MATLAB/SIMULINK. These models are used for system design, control and optimization;
%					\item Applied the models for an optimal design of packed bed thermal energy storage that increases the system cycle efficiency by $\sim$10\%, achieved by optimising the storage materials in the heat storage%, optimal design of radial turbine in CAES applications, system optimization of grid-scale CAES and TES applications, etc.
%					%\item Evaluated system performance by assembling and analysing time-dependent and multi-scale simulation data sets in many case studies of energy systems
%					%\item Improved the system cycle efficiency by $\sim$10\% by optimising the storage materials in the heat storage
%					\end{itemize}

		\item	Ultra-Supercritical (USC) steam power generation technology with Circulating Fluidized Bed (CFB): Combustion, Materials and Modelling
%					\begin{itemize}
%					%\item Created a light-weight super-fluid property data package (MATLAB and Fortran), which is an extendible and computational-efficient tool for academic and industrial partners in the project
%					\item Developed a zone-based dynamic model (in Fortran module compatible with Aspen+) to accurately predict CFD boiler energy distribution with consideration of coupling heat exchange between the free board and the water-wall, validated the model and simulated industrial-scale CFB boiler with parameters collected from industrial partners and Tsinghua University%, increasing the system-level modelling fidelity without negligible computational increases
%					%\item Built and validated a program prototype to simulate industrial-scale CFB boilers with the collected data from industrial partners and the Tsinghua University 
%					%\item Built the super-fluid property data package for Ultra-Supercritical water/steam system
%					%\item Developed heat transfer numerical models of a CFB boiler integrated with water-wall heat exchange
%					\end{itemize}
		\end{itemize}

\item 2012. 9 - 2017. 1  \hspace{2pt} PhD candidate, working on the project \textsl{About Flow} funded by the European commission
		\begin{itemize}
		\item Led the development team of in-house CFD codes (in Fortran) for incompressible flow with discrete adjoint sensitivity/gradient solvers using Automatic Differentiation
%		\begin{itemize}
%		%\item Developed solvers on cell-centred/face-based and node-centred/edge-based data structure
%		%\item Preprocessed mesh data: sorting, listing, searching and tagging elements and calculating geometric information
%		%\item Restructured high-order schemes via node data interpolations
%		%\item Proposed the pseudo-inverse approach for Pressure Schur Complement (non-linear flow system algorithms) to solve incompressible Navier-Stokes Equation
%		%\item Applied matrix preconditioning techniques for solving discretised linear sub systems with data compressed in Compressed Raw Storage (CRS)
%		%Implemented interfaces for data passing through Compressed Raw Storage (CRS)
%%		
%%		\item Applied matrix preconditioning techniques for solving discretised linear sub systems
%% 		\item Applied data post-processing and visualization in Matlab and implemented interfaces (in Fortran) to .vtk output files compatible with ParaView 
%		%\end{itemize}
%		%\item	SIMPLE-like algorithms vs. Pressure Schur Complement (PSC) method theoretical derivation for flow solver development
%		%\begin{itemize}
%		%\item Proposed the pseudo-inverse approach for Pressure Schur Complement (non-linear flow system algorithms) to solve incompressible Navier-Stokes Equation
%		%\item Applied matrix preconditioning techniques for solving discretised linear sub systems
%		%\item Improved solver performance: increased solution accuracy with residuals reduced by 1 order of magnitude and reduced CPU time by $\sim$60\% in a run of both flow and gradient computation (tested on PCs and the HPC cluster)
%		%\item Reduced CPU time by $\sim$60\% in a run of both flow and gradient computation (on PCs and the HPC cluster)
%		%\item Increased the solver robustness for wider applications and compatibility with other post-processing tools%for convergence in wider control parameter space and skewer mesh cases
%		%\item Expanded the solver compatibility to other data post-processing tools for better analysing and visualising data
%		%\end{itemize}
%		\item CAD-based shape optimization of the S-bend air duct in Volkswagen Golf vehicle for reducing pressure drop%by $\sim$20\%
%		%\begin{itemize}
%%		Processed the 3D mesh perturbation using NURBS-based parametrisation with continuity constraints for gradient validation of surface nodes w.r.t. control points
%		%\item Processed the 3D mesh perturbation using in-house CAD tool (NURBS-based parametrisation with continuity constraints) for gradient calculation of surface nodes w.r.t. control points
%%		\item	Developed CAD-based shape optimization driver (in Fortran and Shell script)
%		%\item Optimised the shape of S-bend air duct and achieved pressure loss reduction by $\sim$20\%%analysed fluid dynamics at different inlet speeds%with mesh deformation based on linear elasticity theory
%		%\end{itemize}
%		%Gradient validation on 
%		
%		
%		%\item	Fluid dynamics analysis on  shape optimisation cases 
%		\item	Node-based shape optimisation of the filaments in membrane channels for reducing pressure drop and increasing mass transfer rate
%		%Developed and implemented numerical models for Pressure Retarded Osmosis (PRO) and Reverse Osmosis (RO) membrane process
%					%\begin{itemize}
%					%\item Developed and implemented numerical models for Pressure Retarded Osmosis (PRO) and Reverse Osmosis (RO) membrane process (in OpenFOAM and Fortran codes)
%%					\item Validated the numerical models with experiment data
%					%\item Computed mass transport across the membrane in dual-channel flow using OpenFOAM preprocessing tools
%					
%					%\item Implemented particle swarm optimization algorithm, differential evolutionary optimization algorithm, non-dominated sorting genetic algorithm II (NSGA-II) in MATLAB
%					%\item Applied the MATLAB-based model library and the implemented optimization algorithms to various applications of system optimization, e.g. optimal operation of desalination system, system design of solar powered desalination, etc.
%					%\item Firstly analysed filament surface sensitivities obtained from discrete adjoint computation
%					%\item Designed/Optimised the spacer shape (in Fortran and Shell script) and achieved pressure/energy loss reduction by $\sim$25\% with negligible mass transfer loss					
%					% via gradient-based optimisation driver using discrete adjoint approach and analysed fluid dynamics and filament surface sensitivities
%					%\item Developed a membrane model library (in MATLAB) for simulating the flow and mass transfer of water and salts for system design of solar powered desalination
%					\end{itemize}
%		
		\end{itemize}
		
\item 2009. 9 - 2012. 7  \hspace{2pt} Postgraduate project: Spectral element method for acoustic propagation problem in non-uniform flows
%\begin{itemize} 
%\item Spectral element method for acoustic propagation problem in non-uniform flows
%%		\begin{itemize}
%%		\item Studied Spectral Element Method, the combination of Finite Element and Spectral methods for high-accuracy, multi-scale flow and acoustic coupling computation% for Mathematical Physics Equations
%%		\item	Derived the mathematical description of acoustic propagation in non-uniform flow
%%		\item	Implemented the equation and %group velocity method with 
%%		a high accuracy scheme on the absorbing boundary conditions (in C++)
%%		\end{itemize}
%\end{itemize}

\item 2008. 9 - 2009. 6  \hspace{2pt} Undergraduate project: the design of high flow rate vortex/generative blower
%		\begin{itemize}
%		\item Impeller design based on empirical correlations in literature and 3D model via software ProE
%		\end{itemize}
		
\end{itemize}

\vspace{-12pt}
\section*{Selected Achievements}
\begin{itemize}
\item Successfully led the deployment of local and national energy system modelling tools for multiple LAEP projects, resulting in actionable Net Zero energy plans.
\item Developed and delivered training sessions highlighting the impact of energy system modelling tools in projects.
\item Recognized for effective cross-disciplinary collaboration and project delivery within tight deadlines.
\item Led ESME tool promotion initiatives, collaborating with Marketing and Communications team to produce promotional videos and publish blog content that effectively communicated complex modelling capabilities to stakeholders.
\item Demonstrated effective use of AI tools and shared best practices with team members and the wider company, encouraging adoption of new technologies and enhancing overall productivity.
\item Collaborated with ESC's EDI advocates to support inclusive workplace practices and promote effective teamwork across diverse teams.
\end{itemize}

\vspace{-12pt}
\section*{Selected Publications}
\vspace{-10pt}
Please visit \href{https://raw.githubusercontent.com/guliyolanda/CV/master/YW-papers.pdf} {\tt webpage} for  my publications. 

%\printbibliography[env=numbered+bold, heading=none, sorting=ynt, resetnumbers=true]

%\def\FormatName#1{%
%  \def\myname{Yang Wang}%
%  \edef\name{#1}%
%  \ifx\name\myname
%    \textbf{#1}%
%  \else
%    #1%
%  \fi
%}

%\nobibliography{publications_yw, abbrev}
%\bibliographystyle{abbrv}
%
%\begin{enumerate}
%\item \bibentry{wang18stablisation}
%\item \bibentry{WANG201926}
%\item \bibentry{wang18CAES}
%\item \bibentry{HE20172120}
%\item \bibentry{wang2016computational}
%\item \bibentry{zhang2016geometric}
%\item \bibentry{he2016evaluation}
%\item \bibentry{geng2016research}
%\item \bibentry{akbarzadeh2015fixed}
%\item \bibentry{wang2015stabilisation}
%\item \bibentry{he2015maximum}
%\item \bibentry{he2015stand}
%
%%\item \bibentry{}
%\end{enumerate}


%\vspace{-12pt}
%\section*{Awards and grants}
%\vspace{-10pt}
%\begin{itemize}
%\item 2015 Postgraduate Research Fund (Queen Mary University of London)
%\item 2014 Postgraduate student grant (School of Engineering and Material Science, QMUL)
%\item 2012 Best Postgraduates (Top 10\%)
%\item 2010 Outstanding Postgraduate Student Award (Top 15\%)
%\item 2009 Postgraduate Innovation Fund Scholarship (1st Class, 2/46)
%\item 2009 Best Graduates (Top 10\%)
%\item 2008 \textsl{Fusheng} Industrial Scholarship (1st Class, Top 15\%)
%\end{itemize}
%
%\vspace{-12pt}
%\section*{Teaching and supervising experiences}
%\vspace{-10pt}
%
%\begin{itemize}
%\item	2012 - 2015  \hspace{2pt} Teaching and demonstrating in undergraduate courses:
%			\begin{itemize}
%			\item Heat Transfer and Fluid Mechanics: coursework tutorial
%			\item Mechanics of Fluids: lab demonstration and reports marking
%			\item Computer Aided Engineering in Fluids and Solids: OpenFOAM tutorial
%			\end{itemize}
%
%%		\begin{itemize}
%%		\item	Heat transfer and Fluid Mechanics%: teaching assistant
%%		\item	Mechanics of Fluids%: experiment demonstration and report marking
%%		\item	Computer Aided Engineering in Fluids and Solids%: OpenFOAM tutorial
%%		\end{itemize}
%\item 2012 - 2015 \hspace{2pt} Leader of the segregated flow solver development team:
%		\begin{itemize}
%		\item Mentoring junior researchers with code review and implementation
%		\end{itemize} 
%\item	2009 - 2012  \hspace{2pt} Instructor in Department of Fluid Machinery
%\end{itemize}
%
%% Footer
%%\begin{center}
%%  \begin{footnotesize}
%%    Last updated: \today \\
%%  \end{footnotesize}
%%\end{center}

\end{document}
